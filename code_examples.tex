\usepackage{minted}

\defverbatim[colored]\abba{
    \begin{minted}[fontsize=\scriptsize,
                   framesep=3mm,
                   autogobble
                   ]{pycon}
    >>> a = 1
    >>> b = 1
    >>> a, b = b, a
    \end{minted}
}


\defverbatim[colored]\AstropyUnits{
    \begin{minted}[fontsize=\scriptsize,
                   framesep=3mm,
                   autogobble
                   ]{pycon}
    >>> from astropy import units as u
    >>> distance = 44 * u.imperial.mile
    >>> time = 30 * u.minute
    >>> distance / time
    <Quantity 88.0 mi / h>
    >>> (distance/time).to(u.m/u.s)
    <Quantity 39.33952 m / s>
    >>> (1.21 * u.GW).cgs
    <Quantity 1.21e+16 erg / s>
    >>> 2 * u.m / u.s + 4 * u.m / u.s ** 2
    UnitConversionError: Can only apply 'add' function to quantities
    with compatible dimensions
    \end{minted}
}

\defverbatim[colored]\keV{
    \begin{minted}[fontsize=\scriptsize,
                   framesep=3mm,
                   style=emacs,
                   autogobble
                   ]{pycon}
    >>> kT = 1.2 * units.keV
    >>> kT.to(u.K, equivalencies=u.temperature_energy())
    <Quantity 13925426.47248121 K>
    \end{minted}
}

\defverbatim[colored]\atomic{
    \begin{minted}[fontsize=\scriptsize,
                   framesep=3mm,
                   style=emacs,
                   autogobble
                   ]{pycon}
    >>> from plasmapy.atomic import *
    
    >>> alpha = Particle("He-4++")
    >>> alpha.mass
    <Quantity 6.64465709e-27 kg>
    >>> electron = Particle("e-")
    >>> electron.charge
    <Quantity -1.60217662e-19 C>
    >>> electron.is_category(require={"lepton", "fermion"})
    True
    >>> ~electron  # find antiparticle with invert operator
    Particle("e+")
    \end{minted}
}

\defverbatim[colored]\nuclearreactionenergy{
    \begin{minted}[fontsize=\scriptsize,
                   framesep=3mm,
                   style=emacs,
                   autogobble
                   ]{pycon}
    >>> nuclear_reaction_energy("D + T -> alpha + n").to('MeV')
    <Quantity 17.58932778 MeV>
    \end{minted}
}

\defverbatim[colored]\mathematics{
    \begin{minted}[fontsize=\scriptsize,
                   framesep=3mm,
                   style=emacs,
                   autogobble
                   ]{pycon}
    >>> from plasmapy.mathematics import *
    
    >>> plasma_dispesion_func(0.4 - 0.6j)
    (-2.2969520508628873+2.9148177319960276j)
    
    >>> plasma_dispersion_func_deriv(-1.52+0.47j)
    (0.16587133149822897+0.44587978805935047j)
    
    >>> Fermi_integral(1, 1)
    (1.8062860704447743-0j)
    \end{minted}
}

\defverbatim[colored]\physics{
    \begin{minted}[fontsize=\scriptsize,
                   framesep=3mm,
                   style=emacs,
                   autogobble
                   ]{pycon}
    >>> from plasmapy.physics import *
    
    >>> Debye_length(n_e = 1e15 * u.m ** -3, T_e = 6e6 * u.K)
    <Quantity 0.00534541 m>
    
    >>> inertial_length(5e19 * u.m ** -3, particle='D+')
    <Quantity 0.04553085 m>
    
    >>> upper_hybrid_frequency(0.2 * u.T, n_e = 5e19 * u.m ** -3)
    <Quantity 4.00459419e+11 rad / s>

    >>> B = 2 * u.T
    >>> species = ['e-', 'D+']
    >>> n = [1e18 * u.m ** -3, 1e18 * u.m ** -3]
    >>> omega = 3.7e9 * (2 * pi) * (u.rad / u.s)
    >>> L, R, P = cold_plasma_permittivity_LRP(B, species, n, omega)
    >>> L
    <Quantity 0.63333549>
    >>> R
    <Quantity 1.41512254>
    >>> P
    <Quantity -4.8903104>
    \end{minted}
}

\defverbatim[colored]\classicaltransport{
    \begin{minted}[fontsize=\scriptsize,
                   framesep=3mm,
                   style=emacs,
                   autogobble
                   ]{pycon}
    >>> from plasmapy.transport import *
    >>> T = 1 * u.MK
    >>> n = 5e15 * u.m ** -3
    >>> particles = ('e-', 'p+')
    >>> collision_frequency(T, n, particles)
    <Quantity 443.02775451 Hz>
    >>> coupling_parameter(T, n, particles)
    <Quantity 4.60608476e-06>

    >>> T_e, n_e = 0.6 * u.keV, 1e16 * u.cm ** -3
    >>> T_p, n_p = 0.8 * u.keV, 1e16 * u.cm ** -3
    >>> braginskii = ClassicalTransport(T_e, n_e, T_p, n_p, 'p+')
    >>> braginskii.ion_thermal_conductivity()
    <Quantity 132961.01785222 W / (K m)>
    >>> braginskii.electron_viscosity() # Eq 2.25-2.27 in Braginskii (1965) 
    <Quantity [0.02734206, 0.02733305, 0.02733305, 0., 0.] Pa s>

    \end{minted}
}

\defverbatim[colored]\docstringA{
    \begin{minted}[fontsize=\tiny, 
                   framesep=3mm,
                   style=emacs,
                   autogobble
                   ]{python3}
def plasma_dispersion_func(zeta):
    r"""Calculate the plasma dispersion function

    Parameters
    ----------
    zeta : complex, float, ndarray, or Quantity
        Argument of plasma dispersion function.

    Returns
    -------
    Z : complex, float, or ndarray
        Value of plasma dispersion function.

    Raises
    ------
    TypeError
        If the argument is invalid.
    UnitsError
        If the argument is a Quantity but is
        not dimensionless
    ValueError
        If the argument is not entirely finite

    Notes
    -----
    The plasma dispersion function is defined as:
    
    .. math::
    
    Z(\zeta) = \sqrt{\pi}\int_{-\infty}^{+\infty}dx
               \frac{e^{-x^2}}{x-\zeta}

    where the argument is a complex number
    [fried.conte-1961].

    In plasma wave theory, the plasma dispersion
    function appears frequently when the
    background medium has a Maxwellian distribution
    function.  The argument of this function then
    refers to the ratio of a wave's phase velocity
    to a thermal velocity.

    References
    ----------
    .. [fried.conte-1961]
    Fried, Burton D. and Samuel D. Conte. 1961.
    The Plasma Dispersion Function: The Hilbert
    Transformation of the Gaussian. 

    Examples
    --------
    >>> plasma_dispersion_func(1j)
    0.75787215614131187j
    >>> plasma_dispersion_func(-1.52+0.47j)
    (0.60888889572342553+0.33494583882874029j)
    """
\end{minted}
}
\end{document}